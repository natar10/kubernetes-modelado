En los últimos años, las competiciones de juegos de cartas y juegos de mesa cooperativos han experimentado un auge significativo. 
Títulos como \textit{Magic: The Gathering}, \textit{PokémonTCG} o \textit{Yu-Gi-Oh!} cuentan con
comunidades globales de jugadores que se reúnen regularmente para competir en torneos
organizados.

Actualmente, la organización de estas partidas se realiza a menudo a través de grupos
en redes sociales o aplicaciones de mensajería como WhatsApp. Sin embargo, estas
herramientas no están diseñadas para este fin y no permiten gestionar inscripciones con
plazas limitadas, no ofrecen visibilidad sobre eventos futuros de forma estructurada.
Surgiendo un problema logístico importante ya que no se puede escalar cuando una 
franquicia necesita coordinar eventos en múltiples locales simultáneamente. \\

\textbf{¿Porqué un sistema distribuido es necesario para solucionar el problema?}

Las franquicias de tiendas de juego operan sucursales dispersas en distintas
ciudades y países, y eventos como los \textit{Pre-Release} se celebran simultáneamente
en todas ellas durante el mismo fin de semana, generando picos de demanda donde las
inscripciones pueden multiplicarse por 10x respecto a una semana normal. Un sistema
monolítico no permite escalar cada componente de forma independiente para absorber
estos picos selectivos.

Además, la plataforma dependerá de servicios externos como Google Calendar o SMTP cuya
disponibilidad no está garantizada; si estas operaciones se ejecutaran de forma síncrona
en un monolito, un fallo externo bloquearía la creación de eventos. La separación en
servicios permite aislar estos fallos. A esto se suma que un mismo usuario puede tener
roles distintos según el contexto (jugador, juez, organizador), lo que requiere un
sistema de identidad independiente, y que la información de eventos es consumida
simultáneamente por clientes heterogéneos (web, apps móviles, webs de tiendas, pantallas
informativas), cada uno con patrones de acceso diferentes.

\section{Ejemplos de sistemas similares}

\subsection{Eventlink}
Eventlink es una aplicación de gestión de eventos de DTU, con una arquitectura modular .NET, utilizada por \textit{Wizards of the Coast} para la gestión de torneos oficiales de \textit{Magic:The Gathering}. Sus componentes principales son el \textbf{API}, el sistema de autenticación (\textbf{Auth}), el \textbf{Data Access} y, el más interesante, el servicio \textbf{Crawler}, del que socaba (\textit{crawl}) información de tus eventos a partir de otras Apps, como Facebook o TicketMaster.

Esta, combinado con una arquitectura dirigida por eventos (EDA), permiten una gran rapidez del movimiento y trabajo de los datos dentro del sistema.

\begin{figure}[H]
    \centering
    \includegraphics[width=0.7\linewidth]{Images/enelink_screenshot.png}
    \caption{Captura de pantalla del cliente de Eventlink}
    \label{fig:eventlink}
\end{figure}